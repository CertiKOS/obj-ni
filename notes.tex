%% The first command in your LaTeX source must be the \documentclass command.
\documentclass[acmsmall,screen,review, nonacm]{acmart}

%%
%% \BibTeX command to typeset BibTeX logo in the docs
%%\AtBeginDocument{%
%%  \providecommand\BibTeX{{%
%%    Bib\TeX}}}

\setcopyright{none}

\usepackage{listings}
\usepackage{xcolor}
\usepackage{cmll}
\usepackage{xspace}
\usepackage{pifont}
\newcommand{\todo}[1]{\todo[color=blue!30]{TODO: #1}}


%%%%%%%%%%%%%%%%%%%%%%%%%%%%%%%%%%%%%%%%%%%%%%%%%%%%%%%%%%%
%	Colors
%%%%%%%%%%%%%%%%%%%%%%%%%%%%%%%%%%%%%%%%%%%%%%%%%%%%%%%%%%%
\definecolor{codegreen}{rgb}{0,0.6,0}


%%%%%%%%%%%%%%%%%%%%%%%%%%%%%%%%%%%%%%%%%%%%%%%%%%%%%%%%%%%
%	Math formatting
%%%%%%%%%%%%%%%%%%%%%%%%%%%%%%%%%%%%%%%%%%%%%%%%%%%%%%%%%%%
\newcommand{\mi}[1]{\ensuremath{\mathit{#1}}}
\newcommand{\mr}[1]{\ensuremath{\mathrm{#1}}}
\newcommand{\mt}[1]{\ensuremath{\texttt{#1}}}
\newcommand{\mtt}[1]{\ensuremath{\mathtt{#1}}}
\newcommand{\mf}[1]{\ensuremath{\mathbf{#1}}}
\newcommand{\mk}[1]{\ensuremath{\mathfrak{#1}}}
\newcommand{\mc}[1]{\ensuremath{\mathcal{#1}}}
\newcommand{\ms}[1]{\ensuremath{\mathsf{#1}}}
\newcommand{\mb}[1]{\ensuremath{\mathbb{#1}}}
\newcommand{\msc}[1]{\ensuremath{\mathscr{#1}}}

\newcommand{\isdef}[0]{\ensuremath{\mathrel{\overset{\makebox[0pt]{\mbox{\normalfont\tiny\sffamily def}}}{=}}}}

\newcommand{\lift}[1]{\ensuremath{\lceil#1\rceil}}

% http://tex.stackexchange.com/questions/5502/how-to-get-a-mid-binary-relation-that-grows
\newcommand{\relmiddle}[1]{\mathrel{}\middle#1\mathrel{}}

\DeclareMathOperator\mydefsym{\ensuremath{\iangleq}}
\newcommand\bnfdef{\ensuremath{\mathrel{::=}}}

%%%%%%%%%%%%%%%%%%%%%%%%%%%%%%%%%%%%%%%%%%%%%%%%%%%%%%%%%%%
%	Math shortcuts
%%%%%%%%%%%%%%%%%%%%%%%%%%%%%%%%%%%%%%%%%%%%%%%%%%%%%%%%%%%

\newcommand{\OB}[1]{\ensuremath{\overline{#1}}}
\newcommand{\llb}{\llbracket}
\newcommand{\rrb}{\rrbracket}
\newcommand{\lla}{\mathopen{\ll}}
\newcommand{\rra}{\mathclose{\gg}}
\newcommand{\ra}{\rightarrow}
\newcommand{\Ra}{\Rightarrow}
\newcommand{\la}{\leftarrow}
\newcommand{\La}{\Leftarrow}
\newcommand{\Da}[1]{\ensuremath{\Downarrow^{#1}}}
\newcommand{\Ua}[1]{\ensuremath{\Downarrow^{#1}}}

\newcommand{\myset}[2]{\ensuremath{\left\{#1 ~\relmiddle|~ #2\right\}}}
\newcommand{\partof}[1]{\ensuremath{\mc{P}(#1)}}

\newcommand{\compstatus}[0]{\ensuremath{\mc{CS}}}
\newcommand{\compending}[0]{\ensuremath{\mc{CE}}}

\newcommand{\divr}[0]{\ensuremath{\Uparrow}\xspace}

\newcommand{\term}[0]{\ensuremath{{\Downarrow}}\xspace}




%%%%%%%%%%%%%%%%%%%%%%%%%%%%%%%%%%%%%%%%%%%%%%%%%%%%%%%%%%%
%	Trace Symbols
%%%%%%%%%%%%%%%%%%%%%%%%%%%%%%%%%%%%%%%%%%%%%%%%%%%%%%%%%%%
\newcommand{\mycode}[1]{\texttt{#1}}
\newcommand{\tracecat}{\ensuremath{\bullet}}
\newcommand{\abs}[1]{\ensuremath{|#1|}}
\newcommand{\directprod}{\ensuremath{~\&~}}
\newcommand{\coherent}[1]{\ensuremath{\coh_{#1}}}
\newcommand{\unit}{\mycode{unit}}


%%
%% end of the preamble, start of the body of the document source.
\begin{document}

%%
%% The "title" command has an optional parameter,
%% allowing the author to define a "short title" to be used in page headers.
\title{Object-based Information Flow Policies}

%%
%% The "author" command and its associated commands are used to define
%% the authors and their affiliations.
%% Of note is the shared affiliation of the first two authors, and the
%% "authornote" and "authornotemark" commands
%% used to denote shared contribution to the research.
\author{Anitha Gollamudi}
%%
%% The abstract is a short summary of the work to be presented in the
%% article.
\begin{abstract}
 Object-based approach to information flow policies such as noninterference.
\end{abstract}


%%
%% This command processes the author and affiliation and title
%% information and builds the first part of the formatted document.
\maketitle

\section{Introduction}


\section{Definitions}

\begin{definition}[Principals]
Confidentiality principals  $\bot$ and $\top$ are  the least and most privileged principals, respectively.
\end{definition}

Let O be some class defined as:
\begin{lstlisting}[language=C++, commentstyle=\color{codegreen}, keywordstyle=\color{magenta}]
  class O {
    //State
    field f1;  
    field f2;  

    //M: list of methods
    R1  m1(A1); 
    R2  m2(A2);
    ...
    Rn  mn(An);
    
  }
\end{lstlisting}

\begin{definition}[Signature of a Object]
\[
\classpart  = \{ m_1: \vec{A_1} \to R_1 \dots  m_n: \vec{A_n} \to R_n \}
\]
\end{definition}

\begin{definition}[Coherent Space]
  Coherent space generated from signature \classpart is defined as
  \begin{itemize}
  \item $\abs{\classpart} = \{ m_i(\vec{v}).v' | \forall i. \vec{v} \in \vec{A_i}, v' \in B_i  \}$
    \item $ m_i(\vec{v_i}).v'_i \coherent{\classpart} m_j(\vec{v_j}).v'_j$ \isdef $m_i(\vec{v_i}) = m_j(\vec{v_j}) \implies v'_i = v'_j $
  \end{itemize}
  
\end{definition}


\begin{definition}[Object Space]
  Given a coherent space \classpart, the object space associated with \classpart is the coherence space $\dagger \classpart$ such that
  \begin{enumerate}
  \item \abs{\dagger \classpart} = $\abs{\classpart}^*$
    \item $\langle a_1, a_2 \dots a_m \rangle $ \coherent{\dagger \classpart} $\langle b_1, b_2 \dots b_n \rangle $ iff   $\forall i \in \text{min}(m, n)$.  $\langle a_1, a_2 \dots a_i \rangle $ =  $\langle b_1, b_2 \dots b_n \rangle \implies$ $a_i \coherent{\classpart} b_i$.
  \end{enumerate}
\end{definition}



\begin{definition}[Trace]
A trace $t$ of an object \classpart is a sequence of events drawn from \abs{\classpart}. In other words, $t \in \abs{\dagger \classpart}$.
\end{definition}





Let $L$ and $H$ represent the partitioning of the methods such that $\classpart = L \uplus H$. Intuitively, if $\classpart = \{ l_1 \dots l_i, h_1, \dots h_j \}$,  $L = \{ l_1 \dots l_i\}$ and $H = \{ h_1 \dots h_j \}$ then an attacker representing public confidentiality level ($\bot$) can  invoke only methods from $L$ whereas a privileged principal ($\top$) can invoke either methods from either $L$ or $H$. 

\begin{definition}[Direct Product]
Let $L$ and $H$ be the coherent spaces, then the direct product $(L \directprod H)$ \dots
\end{definition}


\begin{definition}[Trace Projection]
  Let $\classpart = L \uplus H$. Given a trace $t \in \abs{\dagger (L \directprod H)}$, define projection of $t$ w.r.t $L$ as a trace  \proj{t}{L} $\in \abs{\dagger L}$.
\end{definition}


\begin{definition}[Pure Noninterference for a single object]\label{def:pureni}
  Let $\classpart = L \directprod H$. Given a prefix-closed clique $V \in \dagger(L \directprod H)$, we say $L$ does not depend on (or non-interfering with respect to) $H$ if:
  \[
  \forall t_1, t_2 \in V. \proj{t_1}{ L} =  \proj{t_2}{L} \implies \forall e \in \abs{L}. t_1  \tracecat e \in V \iff t_2 \tracecat e \in V \text{ \textcolor{red}{(must $e$ be exactly the same in both traces?)}}
  \]
\end{definition}

However, Definition \ref{def:pureni} does not  admit  relaxed noninterference.
Consider the below Example~\ref{ex1} where \mycode{avgSalary}  simply returns the average of all salaried employees. Note that this is a declassification event as the result is a function of secret information.


\begin{example} \label{ex1}
  Let $O$ be some database object with the signature:
  \begin{align*}
    O = & \{ \\
    & \mycode{int} \mycode{ employees[]} \\
    \\
    & \mycode{val } \mycode{getSalary}: \mycode{id} \to \mycode{int} \\
    & \funcdef{\mycode{getSalary}}{\mycode{id}} \\
    & \qquad \qquad \{ \mycode{return employees[id]} \} \\
    \\
    & \mycode{val } \mycode{putSalary}: \mycode{id} \times \mycode{int} \to \unit \\
    & \funcdef{\mycode{putSalary}}{ \mycode{id, sal}} \\
    & \qquad \qquad   \{ \mycode{employees[id]} = \mycode{sal}  \} \\
    \\
    & \mycode{val } \mycode{avgSalary}: \unit \to \mycode{int} \\
    & \funcdef{\mycode{avgSalary}}{\unit} \\
    & \qquad \qquad \{ \mycode{return sum(employees)} \} \\
    & \}
  \end{align*}
  Suppose $L = \{ \text{avgSalary} \}$ and $H = \{\text{getSalary}, \text{putSalary} \}$. Intuitively, the  attacker (with confidentiality level $\bot$) can  invoke  methods from $L$ but not from $H$.
  \end{example}

\begin{enumerate}

%% \item \textbf{[Basic Case:]} Let $t_1 = \text{putSalary}(1, 5000).()$ and  $t_2 = \text{putSalary}(2, 5000).()$.
%% Then $t_1\downharpoonright_L = t_2\downharpoonright_L$. The only interesting case is when $e = \text{avgSalary}().5000$.
%% Here we have that $t_1 \tracecat e \in V \iff t_2 \tracecat e \in V$.

\item \textbf{[Declassification Event:]}
If $t_1 = \mycode{putSalary}().5000$ and  $t_2 = \mycode{putSalary}().6000$ and $e = \mycode{avgSalary}().5000$, then $t_1\downharpoonright_L = t_2\downharpoonright_L$ but  $t_1 \tracecat e \in V \not \Rightarrow t_2 \tracecat e \in V$. Thus, as per the definition, $L$ depends on $H$ or $L$ is interfering with $H$ (due to declassification).

\item \textbf{[Inadmissible by Definition~\ref{def:pureni}:]}
A slightly different case,  if $t_1 = \mycode{avgSalary}().5000$ and  $t_2 = \mycode{avgSalary}().6000$, then $t_1\downharpoonright_L \ne t_2\downharpoonright_L$. Definition~\ref{def:pureni} does not consider this case.

\end{enumerate}

We relax Definition \ref{def:pureni} to include observational equivalence  than strict equality of public events.
We also introduce a new partition class $D$, to represent methods that declassify information. Thus, $L$, $D$ and $H$ represent the partitioning of the methods such that $\classpart = L \uplus D \uplus H$.
Note that with this partitioning, the $\bot$ attacker can invoke methods from both $L$ and $D$ but not from $H$.

%% \medskip
%% Given a class definition \classpart, define \emph{trace equivalence} of traces $t_1$ and $t_2$ with respect to a principal $\ell$ as follows:
%% \[
%% \treq{t_1}{t_2}{\ell}{\classpart} \isdef \ob{t_1}{\classpart}{\ell} \obsim{\ell}{\classpart} \ob{t_2}{\classpart}{\ell}
%% \]
%% Trace equivalence captures the  observational equivalence using the


The \emph{observation function}, \func{\obsymbol_{\classpart}}{\mathcal{T} \times \mathcal{P}}{\obsset}, a function from the set of traces  and  principals to the set of observations.
The function is parameterized by security specification of the class that describes the input/output security levels of the methods belonging to the class \classpart.




\begin{example}[Identity as Observation Function] \label{ex2}
  Let $O$ be the class from Example \ref{ex1} such that  $L = \emptyset $, $D = \{ \mycode{avgSalary} \}$ and $H = \{\mycode{getSalary}, \mycode{putSalary} \}$.
  Note that \mycode{avgSalary} is a declassification event since there is a flow of information from 
  \[
  \ob{e}{O}{L} \isdef \begin{cases}
    v'  & \mbox{ if } e = m_i(\vec{v}).v' \in \abs{\classpart.L \directprod \classpart.D} \\
    \emptytr & \mbox{ if } e \in \abs{\classpart.H}
  \end{cases}
  \]
\end{example}



\begin{example}[Summation as Observation Function]  \label{ex3}
  Let $O$ be the class from Example \ref{ex1}.
  %% such that  $L = \emptyset $, $D = \{ \mycode{avgSalary} \}$ and $H = \{\mycode{getSalary}, \mycode{putSalary} \}$.
  %% Note that \mycode{avgSalary} is a declassification event since there is a flow of information from 
  \[
  \ob{e \tracecat t }{O}{L} \isdef \begin{cases}
    s + \ob{t}{O}{L}  & \mbox{ if } e = \mycode{putSalary(eid, s).0} \\
    0 & \mbox{ if } t = \epsilon \\
    \ob{t}{O}{L} & \mbox{ o.w } 
  \end{cases}
  \]
\end{example}

%% The observation function is quite expressive: it can encode the attacker's knowledge as show below.

%% \begin{example}[Expressive  Observation Function] \label{ex4}
%%   Let $O$ be the class from Example \ref{ex1} such that  $L = \emptyset $, $D = \{ \mycode{avgSalary} \}$ and $H = \{\mycode{getSalary}, \mycode{putSalary} \}$.
%%   Note that \mycode{avgSalary} is a declassification event since there is a flow of information from 
%%   \[
%%   \ob{e}{O}{L} \isdef \begin{cases}
%%     v' \mbox{ if } e = m_i(\vec{v}).v' \in \abs{\classpart.L} \\
%%     1 \mbox{ if } e \in \abs{\classpart.D} \\
%%     \emptytr \mbox{ if } e \in \abs{\classpart.H}
%%   \end{cases}
%%   \]
%%   The function overwrites the output of $e$ with $1$ making it effectively a constant function.
%% \end{example}

\begin{definition}[Observational Equivalence \obsim{\ell}{\classpart} ]\label{def:obsim}
Given an object specification \classpart and an observation function \obsymbol,  two traces, $t_1$ and $t_2$ are observationally equivalent w.r.t a principal $\ell$, denoted $t_1 \obsim{\ell}{\classpart} t_2$, if \ob{t_1}{\classpart}{\ell} =  \ob{t_2}{\classpart}{\ell}.
\end{definition}


\textbf{Using the \obsymbol function from Example~\ref{ex3}: }
If $t_1 = \mycode{putSalary}(1, 2000) \tracecat \mycode{putSalary}(2, 3000)$ and  $t_2 = \mycode{putSalary}(1, 4000) \tracecat \mycode{putSalary}(2, 1000)$, then \treq{t_1}{t_2}{L}{\classpart}.
However, if $t_1 = \mycode{putSalary}(1, 5000)$ and  $t_2 = \mycode{putSalary}(1, 6000)$, then   \trneq{t_1}{t_2}{L}{\classpart}.


\begin{definition}[\textcolor{red}{(DEPRECATED)} Relaxed Noninterference for a single object]\label{def:relaxni}
   Let $\classpart = (L \directprod D \directprod H)$. Given a prefix-closed clique $V \in \dagger(L \directprod D \directprod H)$, we define noninterference of $L$ w.r.t $H$ if:
  \begin{align*}
   & 1. \forall t_1, t_2 \in V. \treq{t_1}{t_2}{L}{\classpart}  \implies \forall e \in \abs{L}. t_1  \tracecat e \in V \iff t_2 \tracecat e \in V  \\
   & 2. \forall t_1, t_2 \in V. \treq{t_1}{t_2}{L}{\classpart}  \implies \forall m(\vec{v}).v_i \in \abs{D}. \treq{t_1  \tracecat  m(\vec{v}).v_1}{t_2 \tracecat  m(\vec{v}).v_2}{L}{\classpart}
  \end{align*}
\end{definition}

%% The first requirements ensures that all public events match exactly.
%% The condition $\forall m(\vec{v}).v_i \in \abs{L \directprod D}$ in the second requirement ensures that the declassification event has the same inputs but may have different outputs.


%% Definition~\ref{def:relaxni} can admit declassification events. For example, if $t_1 = \mycode{putSalary}(1, 2000) \tracecat \mycode{putSalary}(1, 3000)$ and  $t_2 = \mycode{putSalary}(1, 4000) \tracecat \mycode{putSalary}(1, 1000)$, then we have that \treq{t_1}{t_2}{L}{\classpart}.
%% If $m().v=\mycode{avgSalary}().5000$ then we also have that \treq{t_1  \tracecat  m().v}{t_2 \tracecat  m().v}{L}{\classpart} satisfying Definition~\ref{def:relaxni}.


\medskip
    {
      \color{blue}
\begin{definition}[Relaxed Noninterference for a single object]\label{def:relaxninew}
   Let $\classpart = (L \directprod D \directprod H)$. Given a prefix-closed clique $V \in \dagger(L \directprod D \directprod H)$, we define noninterference of $L$ w.r.t $H$ if:
   \begin{align*}
     & 1. \forall t_1, t_2 \in V \text{ and }  e \in \abs{L \directprod D}. \treq{t_1}{t_2}{L}{\classpart}  \wedge t_1 \tracecat e \in V \wedge t_2 \tracecat  e \in V \implies \treq{t_1  \tracecat  e}{t_2 \tracecat  e}{L}{\classpart} \\
     & 2. \forall t_1, t_2 \in V \text{ and }  e_1, e_2 \in \abs{H}. \treq{t_1}{t_2}{L}{\classpart}  \wedge t_1 \tracecat e_1 \in V \wedge t_2 \tracecat  e_2 \in V \implies \treq{t_1  \tracecat  e_1}{t_2 \tracecat  e_2}{L}{\classpart}
   \end{align*}
\end{definition}
}

    
\clearpage
\section{NI for Multiple Objects}

\begin{definition}[Noninterference for multiple objects]\label{def:ni2}

  \mytodo{\dots}
  
\end{definition}



%%
%% The next two lines define the bibliography style to be used, and
%% the bibliography file.
\bibliographystyle{ACM-Reference-Format}
\bibliography{bibfile}


\end{document}
\endinput
%%
%% End of file `sample-acmsmall-submission.tex'.
