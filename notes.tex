%% The first command in your LaTeX source must be the \documentclass command.
\documentclass[acmsmall,screen,review, nonacm]{acmart}

%%
%% \BibTeX command to typeset BibTeX logo in the docs
%%\AtBeginDocument{%
%%  \providecommand\BibTeX{{%
%%    Bib\TeX}}}

\setcopyright{none}

\usepackage{listings}
\usepackage{xcolor}
\usepackage{cmll}
\newcommand{\mytodo}[1]{\textcolor{red}{#1}}


%%%%%%%%%%%%%%%%%%%%%%%%%%%%%%%%%%%%%%%%%%%%%%%%%%%%%%%%%%%
%	Colors
%%%%%%%%%%%%%%%%%%%%%%%%%%%%%%%%%%%%%%%%%%%%%%%%%%%%%%%%%%%
\definecolor{codegreen}{rgb}{0,0.6,0}
\newcommand{\cmark}{\ding{51}}
\newcommand{\xmark}{\ding{55}}


%%%%%%%%%%%%%%%%%%%%%%%%%%%%%%%%%%%%%%%%%%%%%%%%%%%%%%%%%%%
%	Math formatting
%%%%%%%%%%%%%%%%%%%%%%%%%%%%%%%%%%%%%%%%%%%%%%%%%%%%%%%%%%%
\newcommand{\mi}[1]{\ensuremath{\mathit{#1}}}
\newcommand{\mr}[1]{\ensuremath{\mathrm{#1}}}
\newcommand{\mt}[1]{\ensuremath{\texttt{#1}}}
\newcommand{\mtt}[1]{\ensuremath{\mathtt{#1}}}
\newcommand{\mf}[1]{\ensuremath{\mathbf{#1}}}
\newcommand{\mk}[1]{\ensuremath{\mathfrak{#1}}}
\newcommand{\mc}[1]{\ensuremath{\mathcal{#1}}}
\newcommand{\ms}[1]{\ensuremath{\mathsf{#1}}}
\newcommand{\mb}[1]{\ensuremath{\mathbb{#1}}}
\newcommand{\msc}[1]{\ensuremath{\mathscr{#1}}}

\newcommand{\isdef}[0]{\ensuremath{\mathrel{\overset{\makebox[0pt]{\mbox{\normalfont\tiny\sffamily def}}}{=}}}}

\newcommand{\lift}[1]{\ensuremath{\lceil#1\rceil}}

% http://tex.stackexchange.com/questions/5502/how-to-get-a-mid-binary-relation-that-grows
\newcommand{\relmiddle}[1]{\mathrel{}\middle#1\mathrel{}}

\DeclareMathOperator\mydefsym{\ensuremath{\iangleq}}
\newcommand\bnfdef{\ensuremath{\mathrel{::=}}}

%%%%%%%%%%%%%%%%%%%%%%%%%%%%%%%%%%%%%%%%%%%%%%%%%%%%%%%%%%%
%	Math shortcuts
%%%%%%%%%%%%%%%%%%%%%%%%%%%%%%%%%%%%%%%%%%%%%%%%%%%%%%%%%%%

\newcommand{\OB}[1]{\ensuremath{\overline{#1}}}
\newcommand{\llb}{\llbracket}
\newcommand{\rrb}{\rrbracket}
\newcommand{\lla}{\mathopen{\ll}}
\newcommand{\rra}{\mathclose{\gg}}
\newcommand{\ra}{\rightarrow}
\newcommand{\Ra}{\Rightarrow}
\newcommand{\la}{\leftarrow}
\newcommand{\La}{\Leftarrow}
\newcommand{\Da}[1]{\ensuremath{\Downarrow^{#1}}}
\newcommand{\Ua}[1]{\ensuremath{\Downarrow^{#1}}}

\newcommand{\myset}[2]{\ensuremath{\left\{#1 ~\relmiddle|~ #2\right\}}}
\newcommand{\partof}[1]{\ensuremath{\mc{P}(#1)}}

\newcommand{\compstatus}[0]{\ensuremath{\mc{CS}}}
\newcommand{\compending}[0]{\ensuremath{\mc{CE}}}

\newcommand{\divr}[0]{\ensuremath{\Uparrow}\xspace}

\newcommand{\term}[0]{\ensuremath{{\Downarrow}}\xspace}
\newcommand{\func}[3]{\ensuremath{#1 : #2 \to #3 }}
\newcommand{\funcdef}[2]{\ensuremath{#1 (#2) = }}



%%%%%%%%%%%%%%%%%%%%%%%%%%%%%%%%%%%%%%%%%%%%%%%%%%%%%%%%%%%
%	Trace Symbols
%%%%%%%%%%%%%%%%%%%%%%%%%%%%%%%%%%%%%%%%%%%%%%%%%%%%%%%%%%%
\newcommand{\mycode}[1]{\texttt{#1}}
\newcommand{\tracecat}{\ensuremath{\triangleright}}
\newcommand{\abs}[1]{\ensuremath{|#1|}}
\newcommand{\directprod}{\ensuremath{~\&~}}
\newcommand{\coherent}[1]{\ensuremath{\coh_{#1}}}
\newcommand{\unit}{\mycode{unit}}
\newcommand{\bool}{\mycode{bool}}
\newcommand{\obsymbol}{\ensuremath{\mathbb{O}}\xspace}
\newcommand{\ob}[3]{\ensuremath{\obsymbol_{#2}(#1, #3)}}
\newcommand{\treq}[4]{\ensuremath{#1 \approx^{#4}_{#3} #2}}
\newcommand{\trneq}[4]{\ensuremath{#1 \not \approx^{#4}_{#3} #2}}
\newcommand{\classpart}{\ensuremath{\mathcal{M}}\xspace}
\newcommand{\emptytr}{\ensuremath{\epsilon}}
\newcommand{\obsim}[2]{\ensuremath{\approx^{#2}_{#1}}\xspace}
\newcommand{\true}{\ensuremath{\mathbb{T}}}
\newcommand{\proj}[2]{\ensuremath{#1{\downharpoonright_{#2}}}}
\newcommand{\obsset}{\ensuremath{\mathit{Obs}}}
\newcommand{\event}[2]{\ensuremath{{#2}^{#1} }}
\newcommand{\upperbound}[1]{\ensuremath{{#1}^{\triangle}}}
\newcommand{\downgrade}[2]{ \ensuremath{ #1 \Downarrow_{#2}  } }


%%
%% end of the preamble, start of the body of the document source.
\begin{document}

%%
%% The "title" command has an optional parameter,
%% allowing the author to define a "short title" to be used in page headers.
\title{Object-based Information Flow Policies}

%%
%% The "author" command and its associated commands are used to define
%% the authors and their affiliations.
%% Of note is the shared affiliation of the first two authors, and the
%% "authornote" and "authornotemark" commands
%% used to denote shared contribution to the research.
\author{Anitha Gollamudi}
%%
%% The abstract is a short summary of the work to be presented in the
%% article.
\begin{abstract}
 Object-based approach to information flow policies such as noninterference.
\end{abstract}


%%
%% This command processes the author and affiliation and title
%% information and builds the first part of the formatted document.
\maketitle

\section{Introduction}


\section{Definitions}

\begin{definition}[Principals]
Confidentiality principals  $\bot$ and $\top$ are  the least and most privileged principals, respectively.
\end{definition}

Let O be some class defined as:
\begin{lstlisting}[language=C++, commentstyle=\color{codegreen}, keywordstyle=\color{magenta}]
  class O {
    //State
    field f1;  
    field f2;  

    //M: list of methods
    R1  m1(A1); 
    R2  m2(A2);
    ...
    Rn  mn(An);
    
  }
\end{lstlisting}

\begin{definition}[Signature of a Object]
\[
O = \{ m_1: \vec{A_1} \to R_1 \dots  m_n: \vec{A_n} \to R_n \}
\]
\end{definition}

\begin{definition}[Coherent Space]
  Coherent space generated from signature $O$ is defined as
  \begin{itemize}
  \item $\abs{O} = \{ m_i(\vec{v}).v' | \forall i. \vec{v} \in \vec{A_i}, v' \in B_i  \}$
    \item $ m_i(\vec{v_i}).v'_i \coherent{O} m_j(\vec{v_j}).v'_j$ \isdef $m_i(\vec{v_i}) = m_j(\vec{v_j}) \implies v'_i = v'_j $
  \end{itemize}
  
\end{definition}


\begin{definition}[Object Space]
\dots
\end{definition}

\begin{definition}[Direct Product]
\dots

\end{definition}


\begin{definition}[Trace]
\dots

\end{definition}


\begin{definition}[Projection]
\dots

\end{definition}


Let $L$ and $H$ represent the partitioning of the methods such that $M = L \uplus H$. Intuitively, if $M = \{ l_1 \dots l_i, h_1, \dots h_j \}$,  $L = \{ l_1 \dots l_i\}$ and $H = \{ h_1 \dots h_j \}$ then an attacker representing public confidentiality level ($\bot$) can  invoke only methods from $L$ whereas a privileged principal ($\top$) can invoke either methods from either $L$ or $H$. 

\begin{example} \label{ex1}
  Let $O$ be some database object with the signature:
  \begin{align*}
    O = & \{ \\
    & \text{getSalary}: \mycode{id} \to \mycode{int}, \\
    & \text{putSalary}: \mycode{id} \times \mycode{int} \to \unit \\
    & \text{avgSalary}: \unit \to \mycode{int} \\
    & \}
  \end{align*}
  Then $L = \{ \text{avgSalary} \}$ and $H = \{\text{getSalary}, \text{putSalary} \}$. Intuitively, the  attacker (with confidentiality level $\bot$) can  invoke  methods from $L$ but not from $H$.
  \end{example}

\begin{definition}[Noninterference for a single object]\label{def:ni}
  Given a prefix-closed clique $V \in \dagger(L \directprod H)$, we say $L$ does not depend on (or non-interfering with respect to) $H$ if:
  \[
  \forall t_1, t_2 \in V. t_1\downharpoonright_L = t_2\downharpoonright_L \implies \forall e \in \abs{V}. t_1 \tracecat e \in V \iff t_2 \tracecat e \in V
  \]
\end{definition}


However, it is not clear if Definition \ref{def:ni} can  admit  relaxed noninterference in a meaningful way.
Consider Example~\ref{ex1}.

\begin{enumerate}

\item Let $t_1 = \text{putSalary}(1, 5000).()$ and  $t_2 = \text{putSalary}(2, 5000).()$.
Then $t_1\downharpoonright_L = t_2\downharpoonright_L$. The only interesting case is when $e = \text{avgSalary}().5000$.
Here we have that $t_1 \tracecat e \in V \iff t_2 \tracecat e \in V$.

\item
However,  if $t_1 = \text{putSalary}().5000$ and  $t_2 = \text{putSalary}().6000$ and $e = \text{avgSalary}().5000$, then we have that $t_1 \tracecat e \in V \not \implies t_2 \tracecat e \in V$, a contradiction.
Thus, $L$ does depend on $H$.

\item
A slightly different case,  if $t_1 = \text{avgSalary}().5000$ and  $t_2 = \text{avgSalary}().6000$, then $t_1\downharpoonright_L \ne t_2\downharpoonright_L$. Definition~\ref{def:ni} does not consider this case.

\end{enumerate}




%%
%% The next two lines define the bibliography style to be used, and
%% the bibliography file.
\bibliographystyle{ACM-Reference-Format}
\bibliography{bibfile}


\end{document}
\endinput
%%
%% End of file `sample-acmsmall-submission.tex'.
